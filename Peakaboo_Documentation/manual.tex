\documentclass[article,11pt]{report}

%Images
\usepackage[pdftex]{graphicx}

%In-document links
\usepackage[linktoc=all, colorlinks=true, linkcolor=blue]{hyperref}

%text representing commands or in-program text
\newcommand{\command}[1]{\texttt{#1}}
%Place an icon inline with the text
\newcommand{\icon}[1]{\includegraphics[height=1em]{icons/#1.png}}
%Place and icon and command text together to represent a button
\newcommand{\button}[2]{\ \command{\mbox{\icon{#1} #2}}}
%Keyboard shortcut combining a modifier key and another key
\newcommand{\shortcut}[2]{\command{#1 + #2}}
%Menu chain
\newcommand{\menu}[0]{$\rightarrow$}

%placing a screenshot as a figure
\newcommand{\screenshot}[2]{%
\begin{figure}[h!]
\centering\includegraphics[width=0.85\textwidth]{figures/#1.png}
\caption{#2}
\end{figure}
}


%Table of Contents macros
\newcommand{\tocchapter}[1]{\chapter*{#1}\addcontentsline{toc}{chapter}{#1}}
\newcommand{\tocsection}[1]{\section*{#1}\addcontentsline{toc}{section}{#1}}
\newcommand{\tocsubsection}[1]{\subsection*{#1}\addcontentsline{toc}{subsection}{#1}}
\newcommand{\tocsubsubsection}[1]{\subsubsection*{#1}\addcontentsline{toc}{subsubsection}{#1}}

\begin{document}

\tableofcontents

\tocchapter{Getting Started}


\tocsection{Opening Data}
Select \button{document-open}{Open} from the toolbar or 
\command{File \menu \button{document-open}{Open}} from the menu to locate your XRF scan files. 
You can browse your directories to find your data files. If all scanpoint data are 
contained in a single file, select it to open the file in the data directory. If 
there are many individual ``scanpoint'' files, locate them in the directory and 
use \shortcut{Control}{A} to select all of them to open.

\screenshot{open-files}{Peakaboo's File Selection Dialog}


\tocsection{Examining the Data}

The \icon{zoom-in} \icon{zoom-out} control bar on the bottom right of the window 
allows you to expand the x-axis (energy scale). You can use the scroll bar above
this area, click and drag on the spectrum itself, to pan back and forth to view 
different energy regions.

\screenshot{zoom-scroll}{Zoom Controls and Horizontal Scroll Bar}


\tocsection{Individual Scans}

The \command{Scan \#} field on the left side, under the spectrum view, allows you 
to view individual spectra in a map. Click on the arrows to move in the desired 
direction or enter a value and press \command{Enter}. This option will be disabled 
if you are viewing the mean average or strongest per channel composites rather than 
individual scans.

\screenshot{scan-number}{Scan Number Selector}


\tocsection{Energy Levels}

The \command{Max Energy} setting on the upper right provides energy calibration. 
If your dataset contains this value, the maximum energy value will be set 
automatically. If not, you will have to enter it manually. This value can be 
tweaked or adjusted by selecting a major element known or suspected to be present 
in the sample, and adjusting the energy range until the associated peak fits properly. 
First add your axes to the spectrum by selecting \command{View \menu\ Axes} from 
the menu. The Y Axis -- \command{Relative Intensity} and X Axis -- \command{Energy (keV)} should now be 
labeled for you.

\screenshot{max-energy}{The Maximum Energy Selector}

On the left side of the screen, under the \command{Peak Fitting} tab, click on the 
drop-down arrow beside the \button{edit-add}{Add Fittings} button and select 
\command{Elemental Lookup}, and a list of elements will become visible, 
listed in increasing atomic number. Find the element you wish to use as a reference, 
and click on the checkbox on the box next to it. A set of peaks will appear in shaded 
red within the spectrum. Adjust the Max Energy (keV) setting to move the red shaded 
peaks to fit these intense peaks in the spectrum. The Max Energy setting should now
be calibrated to your data.

You should double-check this Max Energy calibration by adding another element. Although
\command{Ar} may not be present in your sample, it typically appears in XRF spectra.
Select \command{Ar} from the peak list and the \command{Ar} K-$\alpha$ peak. It should appear 
at 2.95 keV with the accompanying K-$\beta$ peak. If you start adding element lines and the fits
appear off, check your calibration again.

\tocsection{Customizing the View}

The \command{View} menu contains many settings to help you to visualise the 
spectrum: \command{Logarithmic Scale} (usually best), \command{Axes} (shows energy 
and intensity scales), \command{Mean Average of Spectra} is the average of all scans 
(averages each energy channel across all spectra within a map to give you a much 
improved composite spectrum).

You can also add curve fitting markings to your spectrum: 
\command{View \menu\ Curve Fit \menu\ Element Names} to label each curve with the 
name of the element it represents,
\command{View \menu\ Curve Fit \menu\ Markings} showing transition markings (K-, L-lines), 
and \command{View \menu\ Curve Fit \menu\ Heights} for fitting heights (max counts).

Your spectrum, as it appears in the window, can also be saved by selecting 
\button{device-camera}{Export Plot as Image} from the toolbar, 
\command{File \menu\ Export Plot as Image}, or \shortcut{Control}{P}.  
Image format options include \command{PNG}, \command{SVG} or \command{PDF}. You can 
also save the fittings results as a txt file by selecting 
\command{File \menu \button{document-export}{Export Fittings as Text}}.

\tocsection{Saving and Loading Sessions}

You can save your session information, so that next time you want to look at your XRF 
data, you can just load this information (session) and not have to start from the 
beginning! These options are found under \command{File \menu\ \button{document-save}{Save Session}} 
and \command{File \menu\ Load Session}.



\tocchapter{Peak Fitting}

\tocsection{Escape Peaks}

Before doing any peak fitting, you should consider the type of
detector you used to collect the XRF spectra. Features may be present in your
spectra that are due to incoming X-rays interacting with the detector material. For
example, for a Si detector, there is a probability that some of the incoming X-rays
will interact with the Si and kick out Si K-shell electrons, thereby reducing the
incoming X-ray’s measured energy by 1.74 keV. This escape peak is present for the
major elements in the sample.

To include escape peaks in fitting the peaks in your spectra, click on View and then
‘Escape Peaks’ and choose either Silicon or Germanium, depending on the type of
detector you used. If you do not need to fit escape peaks, choose ‘none’.

\tocsection{Selecting Fittings}

There are three different tools which can be used to fit your data. To access these 
options, click on the drop-down arrow next to the \button{edit-add}{Add Fittings} button at 
the top of the peak fitting tab.

\tocsubsection{Element Lookup}

Begin by adding elements from the lowest atomic number end first and working
your way up. It is recommended that you fit the K lines first. L lines for elements
for tungsten and higher should be fit separately from the K lines for lower atomic
number elements, as the K lines will interfere with the L line fittings.

Once you are happy with the fittings, click \button{choose-ok}{OK}, and the
list of fitted elements will appear, with all the elements checked off in the boxes to
the right of the element name. You can still look at the fit by unchecking an
element, and its associated peaks will be removed from the spectrum fitting.
Rechecking the box will put the fitted element peaks back in.

\tocsubsection{Guided Fitting}

If you are unsure of a particular peak, you may want to use the ‘Guided Fitting’
option. There are 2 ways to access the ‘Guided Fitting’. The default option for 
\button{edit-add}{Add Fittings} is ‘Guided Fitting’. Click on \button{edit-add}{Add Fittings} and then click on the
peak you wish to fit and the program will fit the peak with a possible option. Click
\button{choose-ok}{OK} to add the lines to your list. Other possible element lines will also be
displayed in the drop down list. The second way is to choose the ‘Guided Fitting’
option from the drop down list. Then click on the unknown peak and a list of
possible line fittings will appear in the previously empty drop down list. Highlight
the element line you wish to use for the fitting. To edit a fitting, click \button{edit-edit}{Edit} (see the following figure). To add another fitting, click \button{edit-add}{Add Fittings} and
then click on the peak you wish to identify. Once you are happy with the fittings
click on OK to add the element lines to your list.

In the following example, 2 Pb (lead) L lines were fitted at approximately 10.55 and
12.59 keV. Using the ‘guided fitting’ option to search for an element line to fit the
peak at 12.59 keV, also fit the peak at 10.55 keV with a Pb L Line, when Pb L was
chosen from the list. If a peak at 10.55 keV did not exist, then the presence of Pb
would not be likely. Alternatively, As K alpha line overlaps with the Pb L line at
10.55 keV, however, the presence of the Pb L line at 12.59 keV indicates that Pb is
likely present and not As.

\tocsubsection{Summation Peaks}

If you are having trouble finding element lines to fit some peaks in your spectrum,
you may have what are called ‘pile-up peaks’ (referred to as ‘summation’ peaks in
Peakaboo). This is a detector phenomenon. If your sample has a lot of one element
present, then some of the emitted x-rays due to the presence of this element will
impinge on the detector virtually simultaneously and the pulse created and
measured would be the sum of the two x-ray energies.

The XRF spectrum from a tooth is shown below. The spectrum is partially fitted to
show that the peak structure between 7-8 keV is not fully explained by the
presence of Ni. The tooth contains a large amount of Ca therefore one should
investigate the presence of Ca pile-up peaks.

The figure below shows a truncated XRF spectrum from a tooth containing a large
amount of Calcium. Three Ca pile-up peaks would be expected, one from the
doubling of the Ca Kα lines, a second from the addition of the Ca Kα and β lines,
and a third from the doubling of the Ca Kβ peak energies. These Ca pile-up peaks
are indicated in the figure below.

In order to access the summation fitting option, left click on arrow next to the
\button{edit-add}{Add Fittings} button at the top of the peak fitting tab. Remember 
to click on \button{choose-ok}{OK} to add the summation fits to your spectrum. 
Please note that for element lines to appear in the drop down lists, they must 
have already been included in the spectral fit.

The fully fitted spectrum is shown below. Ni K and Cu K lines have been added and
overlap with the Ca pile-up peaks.



\tocchapter{Mapping}

To map the fitted elements, click on \button{map}{Map Fittings} in the toolbar. 
Click on the dropdown arrpw to the right of \button{map}{Map Fittings} to choose either the
fitting area or fitting height for mapping. Only those checked elements in your
``Fitted Elements'' list will be mapped. It is recommended that you check all
elements for mapping as all maps will be included in the mapping results. If your
map is quite large, then it may be better to map one element at a time as the
processing on your desk-top or laptop will take considerable time.

Click on \button{map}{Map Fittings}, a window will pop into view and indicate the
progress of mapping the fitted elements.

Once the mapping is complete, a window will appear with a coloured map and scale
and list of mapped element lines. Scan xml files containing all individual scans will
have the associated meta data so that map widths and heights are automatically
considered. Otherwise, the user must adjust the width and height. All elements will
be checked and represented in the initial map. To get the individual element maps,
uncheck listed elements, leaving the element line map you wish to view. If your
maps have a pixilated appearance, you may want to try interpolating the data and
also adding in some contouring. These features are accessed in the mapping
window as well.

The Zn K map shown in the above figure was scaled to the highest intensity of the
Zn K line fittings from all the individual spectra in the map data set (‘Visible Fittings’
oprion). If you choose the ‘All fittings’ option under the ‘Scale Intensities by:’
heading, then the maps will be scaled to the highest intensity from all element line
fittings from all the individual spectra in the map data set. Thus, a minor element
would show very a low intensity distribution in its map.

Element line maps may be saved by clicking on the \button{document-save}{Save} button under the
element line list in the mapping window. Various formats are available for saving
your element line maps, Pixel Image (png), Vector Image (svg) and PDF format.
Once you have clicked on the \button{document-save}{Save} button, a directory window will appear and
you can browse to select the desired directory and then enter a suitable file name.

\tocsection{Overlays}

The ‘overlay’ option for mapping is chosen from the drop down list within the Map
window, under ‘Mapped Fittings’ (composite, overlay and ratio).

Below: ‘As a group’ option example. The highest intensity from the group of
elements displayed is used to scale the colors for the group. Zn is much less in
intensity than the Ca, thus the presence of Zn is not apparent in the distribution
map.

\tocsection{Ratios}

Mapping of element ratios is also available within Peakaboo. Select Ratio from the
‘Mapped/Fittings’ dropdown list. Select the elements (or element sets) you wish to
ratio and choose the desired colours from the dropdown lists next to the element
line fitting. In the example below, the Zn to Fe ratio is mapped. Areas where the Fe
is strong and there is very little Zn show up as blue.

\tocsection{Errant Data}

Missing or skewed spectra within a map can be removed from the map. An example
is shown below in which one such data point is evident within the map. This
occurred after injection at the synchrotron and the first data point after injection
was not recorded for some reason.

To view the errant data point, make sure you are in ‘Individual Spectrum’ view.
Enter the Index\# you have determined, in the Scan field as shown below. No
spectrum is displayed in the scan window, thus it is immediately evident that the
spectrum is missing for this scan point. Click on the \button{choose-cancel}{Exclude} button as shown below to
flag the scan as bad to exclude it from the data set.

The data point has now been excluded from the map. The errant data point or scan
has been replaced with an averaged one from the surrounding scans.


\tocsection{Plotting Subsections}

The distribution of elements will likely vary across your sample and hence your
maps. Peakaboo allows you to view spectra from selected regions within your map.
Left click and drag the mouse in order to draw a rectangle surrounding the region
you wish to select. Click on the down arrow to plot the selected region.

Another Peakaboo window will open up to display the spectra from the selected
region or subset. The spectra from the selected region or ‘subset’ can then be
processed with Peakaboo.


\tocchapter{Filters}

Filters can be accessed by clicking on the Filters tab and then clicking on 
\button{edit-add}{Add Filters} to display the typed of filters. The filter type heading can be expanded by
clicking on the box to the left of the filter type. Click on the boxes to the left of the filter
type headings in order to list the choices available. Highlight the type of filter you
wish to use and then click on the OK (green check mark) to add the filter. If you want to modify or
check the parameters associated with your filter, click on the \button{misc-preferences}{Settings} button next to the
filter name to bring up the filter settings window. As you change the parameters,
you will see the reflected changes in the spectrum background fitting. Once you are
satisfied with the filter settings, close the window.

The use of filters will obviously alter the spectrum. If you wish to see the raw data
outline, choose this option under the view menu. Individual line fittings can also be
viewed as an option under the view menu.

\tocsection{Example Filter Use}

This example uses the normalize filter to correct for high deadtime artifacts in maps. 
When deadtime is very high, the detector cannot keep up with the incoming x-rays.
For example, low energy x-rays such as those coming from Ar may not get counted
and mapping the distribution of Ar will reveal areas of next to no intensity. The
‘Normalizer Filer’ scales each spectrum so that the intensity at a given channel is
always the same. The ‘Normalizer Filter’ can be used in this case to correct the
spectra to make the counts from the Ar channel to be consistent in all spectra and
scale the other element lines in the spectra appropriately.

Go to the ‘Filter’ tab and add the ‘Normalizer Filter’ found under the ‘Advanced’
filter types. Click on the edit box to bring up the window for editing the settings.
Place the cursor on the Ar peak and read the channel number and value (counts)
from the information found below the spectrum window, and enter them into the
appropriate settings fields. Make sure you are viewing the ‘Mean’ spectrum to get
the average counts (Value) for the Ar. Close the edit window.



\tocsection{Filter Descriptions}

\tocsubsection{Background}

\tocsubsubsection{Polynomial}

This filter attempts to fit a series of parabolic (or higher order single-term) curves 
under the data, with a curve centred at each channel, and attempting to make each curve 
as tall as possible while still staying completely under the spectrum. The union of 
these curves is calculated and subtracted from the original data.

\tocsubsubsection{Brukner}

This filter removes background over several iterations by smoothing the data and 
taking the minimum of the unsmoothed and smoothed data for each channel on each pass.

\tocsubsubsection{Linear Trim}

This filter examins all pairs of points which are n channels apart (ie (1, 10), 
(2, 11) where n = 10). For each pair of points, any signal which exceeds a straight 
line connecting the two points is truncated.

\tocsubsection{Noise}

\tocsubsubsection{Savitsky-Golay}

This filter attempts to remove noise by fitting a polynomial to each
point p0 and its surrounding points p0-n..p0+n, and then taking the value of the
polynomial at point p0. A moving average may be considered a special case of this
filter with a polynomial of order 1.

\tocsubsubsection{Wavelet Low-Pass}

This filter attempts to reduce high-frequency noise by
performing a Wavelet transformation on the spectrum. This breaks the data down
into sections each representing a different frequency range. The high-frequency
regions are then smoothed, and a reverse transform is applied.

\tocsubsubsection{Aggressive Wavelet Low-Pass}

This filter attempts to reduce high-frequency noise by performing a Wavelet 
transformation on the spectrum. This breaks the data down into sections each 
representing a different frequency range. The high-frequency regions are then 
completely removed, and a reverse transform is applied.

\tocsubsubsection{Fourier Low-Pass}

This filter transforms the spectral data with a Fourier
Transformation into a frequency domain. Data from a high frequency range (noise)
is filtered out, while lower frequencies (peaks, background) are passed through.

\tocsubsubsection{Moving Average}

This filter refines the values of each point in a scan by sampling
it and the points around it, and replacing it with an average of the sampled points.

\tocsubsubsection{Spring Smoothing}

This filter operates on the assumption that weak signal should
be smoothed more than strong signal. It treats each pair of points as if they were
connected by a spring. With each iteration, a tension force draws neighbouring
points closer together. The Force Multiplier controls how strongly a pair of elements
are pulled together, and the Force Falloff Rate controls how aggressively stronger
signal is anchored in place, unmoved by spring forces. This prevents peak shapes
from being distorted by the smoothing algorithm.

\tocsubsection{Mathematical}

\tocsubsubsection{Add}

This filter adds a constant value to all points on a spectrum.

\tocsubsubsection{Subtract}

This filter subtracts a constant value to all points on a spectrum.

\tocsubsubsection{Multiply}

This filter multiplies all points on a spectrum by a constant value.

\tocsubsubsection{Derivative}

This filter transforms the data such that each channel represents the
difference between itself and the channel before it.

\tocsubsubsection{Integral}

This filter transforms the data such that each channel represents the sum of
itself and all channels prior to it.


\tocsubsection{Advanced}

\tocsubsubsection{Signal -> Wavelet}

This filter converts spectrum data into a wavelet
representation. This is intended to be used in conjunction with other filters
(especially the 'Filter Partial Spectrum' filter) to perform custom wavelet operations.

\tocsubsubsection{Wavelet -> Signal}

This filter converts a wavelet representation of data back into
spectrum data. This is intended to be used in conjunction with other filters
(especially the 'Filter Partial Spectrum' filter) to perform custom wavelet operations.

\tocsubsubsection{Normalizer}

This scales each spectrum so that the intensity at a specified channel is
always the same across all spectra.

\tocsubsubsection{Filter Partial Spectrum}

This filter allows the application of another filter to a
portion of a spectrum.


\tocsubsection{Programming}

\tocsubsubsection{Java Code}

This filter allows you to create your own filter in the Java programming language.

\tocsubsubsection{JPython Code}

This filter allows you to create your own filter in the Python programming language using JPython.

\end{document}

